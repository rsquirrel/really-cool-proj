\documentclass[12pt,psfig,a4]{article}
\usepackage{geometry}
%\geometry{a4paper}
\geometry{left=28mm,right=28mm,top=21mm,bottom=21mm}

%\gemoetry{verbose,a4paper,tmargin=21mm,bmargin=21mm,lmargin=18mm,rmargin=18mm}
\usepackage{graphics}
\usepackage{setspace}
\newcommand{\Lyx}{L\kern-.1667em\lower.25em\hbox{y}\kern-.125emX\spacefactor1000}
\newcommand\bibname{References}
\singlespacing
\begin{document}
\bibliographystyle{plain} 
\pagestyle{plain} 
\pagenumbering{arabic}
%\rmfamily

\newenvironment{code}
{\hangindent=0.7cm\itshape\sffamily\parskip 7.2pt}
{\parskip 7.2pt}

\title{TML: Language Reference Manual}
\author{
Jiabin Hu (jh3240)\\
Akash Sharma (as4122)\\
Shuai Sun (ss4088)\\
Yan Zou (yz2437)
}
\date{\today}
\maketitle

% Article starts here

%\begin{abstract} 

%\end{abstract} 

%{\bf Keywords:} \\

\section{Introduction}
Tree manipulation(TML) language is user friendly language that is designed to help users program trees. It allows users to create, manipulate and run algorithms on trees. Various existing programming languages make tree operations cumbersome and TML is intended to bridge this gap. In TML everything is of type Tree except primitive data types. Each node of the tree is of type Tree and it has fields associated with it like parent, child etc. Every node of the Tree is root for its subtree. Nodes of the tree can consist of user defined or primitive value types and they can be of any degree. We can refer to any child or parent node of a given node using predefined constructs. TMLprovides methods to perform trivial operations on trees such as tree traversal, node creation or deletion etc.
\textit{This section will be written later.}

\section{Lexical Conventions}


\subsection{Character Set}
TML takes the standard \textbf{ASCII} character set for source codes.

\subsection{Comments}
In TML, there are two ways to make comments. The first style starts with the characters \textbf{/*}, and ends with the characters \textbf{*/}, all contents between which are comments. Note that, just like C-style comments, TML supports only un-nested comments. The second style is inline comments. It starts with the characters \textbf{//}, all contents in the current line after which are regarded as comments.

\subsection{Identifiers}
In TML, an identifier is a string that starts with a letter or an underscore, and consists of a sequece of letters, digits, and underscores. The max length of an identifier is 26 characters. 
All identifiers are case-sensitive in TML.

\subsection{Keywords}
\textit{This section needs discussion.}\\
In TML, the words listed in Table~\ref{keywords} are reserved as keywords, and are not allowed to be used as a user-defined identifier.

\begin{table}[ht]
\begin{center}
\begin{tabular}{| c | c | c | c | c |}
\hline
if & else & type & while & foreach \\
\hline
in & by & do & print &\\
\hline
preorder & inorder & postorder & border &\\
\hline
int & float & char & string & bool  \\
\hline
treetype &  & & & \\
\hline
\end{tabular}
\caption{Keywords in TML}
\label{keywords}
\end {center}
\end{table}

%\subsection{Operators}
%\textit{This section needs discussion.}\\
%TML provides basic operators for arithmetic, comparing, assignment and string operations, which are listed in Table~\ref{basic_operators}.
%
%\begin{table}[ht]
%\begin{center}
%\begin{tabular}{| c | c |}
%\hline
%\textbf{Usage} & \textbf{Operator} \\
%\hline
%Arithmetic & +, -, *, /, \% \\
%\hline
%Comparing & \textgreater, \textgreater=, ==, !=, \textless=, \textless \\
%\hline
%Assignment & =, +=, -=, *=, /=, \%= \\
%\hline
%Sting & ==, sizeof() \\
%\hline
%\end{tabular}
%\caption{Basic Operators in TML}
%\label{basic_operators}
%\end {center}
%\end{table}



\section{Expressions}
In TML, expressions consists of operators and their operands. In this section, we'll begin with defining each operator. After this, we'll define the precedence rules so that the evaluation can be carried out unambigously.

\subsection{Arithmetic Operators}
In TML, arithmetic operators are $+$, $-$, $*$, $/$ and $\%$. $+$ means addition, $-$ means substruction, $*$ means multiplication, $/$ means division and $\%$ means modulation. All of them are binary and left associative. It requires that their operands must be of the same primitive types, and the result will be of the same type.

\subsection{Comparative Operators}
In TML, comparative operators are $>$ (greater than), $<$ (less than), $>=$ (greater than or equal to), $<=$ (less than or equal to), $!=$ (not equal) and $==$ (equal). All of them are binary and left associative. It requires that their operands must be of the same primitive types . The result will be of type bool.

\subsection{Logical Operators}
Logical operators in TML include \&\& (logical and), $||$ (logical or) and ! (logical not). \&\& and $||$ are binary and left-associative. They take two operands of type bool, and return a bool value. ! is unary and appears on the left side of the operand. They type of the operand must be bool and the return type is also bool.

\subsection{Assignment Operators}
TML's assignment operator is =. it's binary and right-associative. The left operand must be an identifier and the right operand must be an expression. It assigns the value of the expression to the identifier, and returns the value of the identifier. This allows chained assignment. In addition, the following short-cut assignment operators are also defined: +=, -=, *=, /=, \%=. They require the expression and identifier of the same type.

\subsection{Parentheses}
In TML, parentheses in expressions are used to overwrite the precedence rules, Expression enclosed by the parentheses will always be calculated first before applying its adjacent operators.

\subsection{Tree Operators}
The highlight of TML is a set of operators that can be applied to trees. They are divided into three categories: tree-building, tree-copying and tree-querying.

\subsubsection{Tree-building operator}
We define $->$ to be the tree building operator. It's binary and right-associative. The left operand is a single tree node, representing the root. The right operand is a list of trees, enclosed by parentheses and seperated by commas, representing the immediate children of the root. For example, a$->$(b, c) means building a tree with a as root and b as its left subtree, and c as its right subtree. This operator returns the newly built tree.

\subsubsection{Tree-copying operators}
We define two operators to copy trees. @ means copying the node and its value only, with all of its children set to null. \$ means creating a new tree, with exactly the same stucture as the original tree. Both of them are unary and appear on the left side of the opearand. They return the new tree.

\subsubsection{Tree-querying operators}
In TML, we define operator \# as follows: \#n means the order of node n among its siblings, starting from 0. For example, if node a has three nodes b, c, d, then \#b, \#c, \#d will return 0, 1, 2, respectively. If n has no parent, \#n will return null.

\subsection{Precedence Rules}
\begin{table}[ht]
\begin{center}
\begin{tabular}{| c | c |}
\hline
1 & () \\
2 & \# \\
3 & @, \$ \\
4 & *, /, \% \\
5 & +, - \\
6 & $<$, $>$, $<=$, $>=$, $!=$, $==$ \\
7 & ! \\
8 & \&\& \\
9 & $||$ \\
10 & =, +=, -=, *=, /=, \%= \\
11 & $->$ \\
\hline
\end{tabular}
\caption{Precedence Rules}
\label{keywords}
\end {center}
\end{table}

\section{Statements}

\subsection{Variable Declaration and Initialization Statement}

\subsubsection{Type Specifiers}

\subsubsection{Declarators}

\subsection{Expression Statements}
The expression statement is the most common one in TML. It consists of an expression and a semicolon at the end. Any expression can be used here. TML will evaluate the expression and ignore the final evaluation result.



\begin{code}
expression;
\end{code}

\subsection{Compound Statements}

\subsection{Conditional Statements}

\subsection{Iterative Statements}

\subsection{Other Statements}

\section{Functions}
Order insensitive

\subsection{Function Definition}

\subsection{Function Return}

\subsection{Function Calls}

\section{Scope}

\section{Language Restrictions}
Like various existing languages like C, Java etc, TML also has some restrictions on length of identifier names, size of basic data types, number of arguments for function calls, number of data fields in a node of type Tree and maximum degree of the node.

\section{Language Formalization}

\section{Conversions to Java}





%\begin{thebibliography}{}

%\end{thebibliography}   

\end{document}
