\documentclass[12pt,psfig,a4]{article}
\usepackage{geometry}
%\geometry{a4paper}
\geometry{left=28mm,right=28mm,top=21mm,bottom=21mm}

%\gemoetry{verbose,a4paper,tmargin=21mm,bmargin=21mm,lmargin=18mm,rmargin=18mm}
\usepackage{graphics}
\usepackage{setspace}
\newcommand{\Lyx}{L\kern-.1667em\lower.25em\hbox{y}\kern-.125emX\spacefactor1000}
\newcommand\bibname{References}
\singlespacing
\begin{document}
\bibliographystyle{plain} 
\pagestyle{plain} 
\pagenumbering{arabic}
%\rmfamily

\title{TML: Language Reference Manual}
\author{
Jiabin Hu (jh3240)\\
Akash Sharma (as4122)\\
Shuai Sun (ss4088)\\
Yan Zou (yz2437)
}
\date{\today}
\maketitle

% Article starts here

%\begin{abstract} 

%\end{abstract} 

%{\bf Keywords:} \\

\section{Introduction}

\textit{This section will be written later.}

\section{Lexical Conventions}

\subsection{Character Set}
TML takes the standard \textbf{ASCII} character set for source codes.

\subsection{Comments}
In TML, there are two ways to make comments. The first style starts with the characters \textbf{/*}, and ends with the characters \textbf{*/}, all contents between which are comments. Note that, just like C-style comments, TML supports only un-nested comments. The second style is inline comments. It starts with the characters \textbf{//}, all contents in the current line after which are regarded as comments.

\subsection{Identifiers}
In TML, an identifier is a string that starts with a letter or an underscore, and consists of a sequece of letters, digits, and underscores. The max length of an identifier is 26 characters. 
All identifiers are case-sensitive in TML.

\subsection{Keywords}
\textit{This section needs discussion.}\\
In TML, the words listed in Table~\ref{keywords} are reserved as keywords, and are not allowed to be used as a user-defined identifier.

\begin{table}[ht]
\begin{center}
\begin{tabular}{| c | c | c | c | c |}
\hline
if & else & type & while & foreach \\
\hline
in & by & do & print &\\
\hline
preorder & inorder & postorder & border &\\
\hline
int & float & char & string & bool  \\
\hline
treetype &  & & & \\
\hline
\end{tabular}
\caption{Keywords in TML}
\label{keywords}
\end {center}
\end{table}

\subsection{Operators}
\textit{This section needs discussion.}\\
TML provides basic operators for arithmetic, comparing, assignment and string operations, which are listed in Table~\ref{basic_operators}.

\begin{table}[ht]
\begin{center}
\begin{tabular}{| c | c |}
\hline
\textbf{Usage} & \textbf{Operator} \\
\hline
Arithmetic & +, -, *, /, \% \\
\hline
Comparing & \textgreater, \textgreater=, ==, !=, \textless=, \textless \\
\hline
Assignment & =, +=, -=, *=, /=, \%= \\
\hline
Sting & ==, sizeof() \\
\hline
\end{tabular}
\caption{Basic Operators in TML}
\label{basic_operators}
\end {center}
\end{table}




%\begin{thebibliography}{}

%\end{thebibliography}   

\end{document}
